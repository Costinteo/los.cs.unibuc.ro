\documentclass[a4paper,12pt]{article}
\usepackage[utf8]{inputenc}
\usepackage{amsmath, amssymb,amscd}
\usepackage{amsthm}
\usepackage{fullpage}
\usepackage{hyperref}
\usepackage{mathrsfs}
 
\usepackage[utf8]{inputenc}
\usepackage[T1]{fontenc}

\newcommand{\AiwX}{\mathcal{A}_i^\omega[X,d]}
\def\N{{\mathbb N}}
\newcommand{\TX}{{\bf T}^X}
\newcommand{\sL}{\mathscr{L}}
\newcommand{\ul}{\underline}
\newcommand{\ra}{\rightarrow}
\newcommand{\Tw}{{\bf 0}}
\newcommand{\cL}{\mathcal{L}}
\newcommand{\cS}{\mathcal{S}}

%opening
\title{Logical Metatheorems for Metric Spaces I}
\author{Speaker: Daniel Ivan, IMAR}
\date{March 21, 2013}
\thispagestyle{empty}
\begin{document}
\maketitle
\thispagestyle{empty}

Given a set $X$ and symbols $b_X$ and $d_X$, we present the theory $\AiwX$, an extension of the weakely extensional 'Heyting' arithmetic theory that also contains the axiom of choice principle.
$\AiwX$ is defined over the set of all finite types generated by the ground types $\N$ and $X$, denoted by $\TX$.

The goal of this seminar is to prove, using the monotone functional interpretation, the following logical metatheorem.\newline
Let $\sigma, \rho, \tau$ be types of $\TX$, such that $\sigma$ is a type of degree $1$, $\rho$ is a type of degree $(\cdot, 0)$, and $\tau$ is a type of degree $(\cdot, X)$.
Let $s^{\rho \sigma}$ be a closed term of $\AiwX$, $B_\forall(x^\sigma, y^\rho, z^\tau, u^{\Tw} )$ be a $\forall$-formula with only $x, y, z, u$ free, and $C(x^\sigma, y^\rho, z^\tau, v^{\Tw} )$ a general formula with only $x, y, z, v$ free.
Both $B$ and $C$ are formulas of $\sL(\AiwX)$.
If 
\[
\AiwX + \Delta \vdash \,\forall x^\sigma \,\forall y\leq_\rho sx \,\forall z^\tau (\forall u^{\Tw}  \,B_\forall (x, y, z, u) \ra \exists v^{\Tw}  \, C(x, y, z, v)),
\]
then one can extract a function $\Phi: \cS_\sigma \times \N \ra \N$, such that for all $x\in \cS_\sigma$ and all $b\in \N$,
\[
\forall x^\sigma, y\leq_\rho sx \,\forall z^\tau [\forall u\leq_{\Tw}  \Phi(x,b) \,B_\forall (x, y, z, u) \ra \exists v \leq_{\Tw}  \Phi(x, b) \,C(x, y, z, v)]
\]
holds in any nonempty metric space whose metric is bounded by $b\in\N$ and that satisfies $\Delta$, where 
\[
 \Delta:\equiv \{\forall \ul{a}^{\ul{\delta}} \, \exists \ul{b} \leq_{\ul{\sigma}} \ul{r}\,\ul{a}\, \forall \ul{c}^{\ul{\gamma}}\, F_0(\ul{a}, \ul{b}, \ul{c})\},
\]
and $F_0$ is a quantifier-free formula of $\cL(\AiwX)$.
The theorem still holds if we add to $\AiwX$:
\begin{itemize}
 \item Independence of premise principle, for universal formulas:
\[
  {\rm IP}_\forall^\omega :\equiv (\,\forall \ul{x}A_0(\ul{x}) \ra \exists y^\rho B(y)) \ra \exists y^\rho(\,\forall \ul{x} A_0(\ul{x}) \ra B(y)),
\]
where $y$ is not free in $\,\forall \ul{x} A_0(\ul{x})$ and $A_0$ is a quantifier free formula.
\item Markov Principle:
\[
 {\rm M}^{\ul{\rho}} :\equiv \lnot \lnot \exists \ul{x}^{\ul{\rho}} A_0(\ul{x}) \ra \exists \ul{x}^{\ul{\rho}} A_0(\ul{x}),
\]
where $A_0$ is an arbitrary quantifier-free formula.
\end{itemize}
The theorem also holds for tuples of variables $\ul{x}, \ul{y}, \ul{z}, \ul{u}, \ul{v}$, of appropriate types.

\end{document}
