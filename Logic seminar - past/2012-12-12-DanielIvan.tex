\documentclass[a4paper,12pt]{article}
\usepackage[utf8]{inputenc}

%opening
\title{Logic Seminar: Proof Interpretations I}
\author{Speaker: Daniel Ivan, IMAR}
\date{December 12, 2012}
\begin{document}

\maketitle
This is the first of a series of talks on proof interpretations, following the book \emph{U, Kohlenbach, ``Applied Proof Theory. Proof Interpretations and their Use in Mathematics''}.

We will introduce the logic in which we are going to work,  Extensional Heyting Arithmetic in all finite types (${\bf E-HA}^\omega$, for short), which is a many-sorted version of Intuitionistic logic (a sort for each finite type), extended with Heyting arithmetic.
We will start our study of proof interpretation with \emph{Modified realizability}. This notion originates from  Georg Kreisel's works in 1959, in which he intended to give a consistency proof for the system Heyting Arithmetic in all finite types (${\bf HA}^\omega$, for short), and accordingly, he defined a straightforward extension of Kleene's realizability to this typed system. 
\end{document}
